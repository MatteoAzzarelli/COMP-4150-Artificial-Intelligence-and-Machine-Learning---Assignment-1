\documentclass[a4 paper]{article}
% Set target color model to RGB
\usepackage[inner=2.0cm,outer=2.0cm,top=2.5cm,bottom=2.5cm]{geometry}
\usepackage{setspace}
\usepackage[rgb]{xcolor}
\usepackage{verbatim}
\usepackage{subcaption}
\usepackage{amsgen,amsmath,amstext,amsbsy,amsopn,tikz,amssymb,tkz-linknodes}
\usepackage{fancyhdr}
\usepackage[colorlinks=true, urlcolor=blue,  linkcolor=blue, citecolor=blue]{hyperref}
\usepackage[colorinlistoftodos]{todonotes}
\usepackage{rotating}
\usepackage{hyperref}   % link web
%\usetikzlibrary{through,backgrounds}
\hypersetup{%
pdfauthor={Matteo Azzarelli},%
pdftitle={Homework},%
pdfkeywords={latex, Auditing},%
pdfcreator={PDFLaTeX},%
pdfproducer={PDFLaTeX},%
}
%\usetikzlibrary{shadows}
\usepackage{booktabs}
\newcommand{\ra}[1]{\renewcommand{\arraystretch}{#1}}

\newtheorem{thm}{Theorem}[section]
\newtheorem{prop}[thm]{Proposition}
\newtheorem{lem}[thm]{Lemma}
\newtheorem{cor}[thm]{Corollary}
\newtheorem{defn}[thm]{Definition}
\newtheorem{rem}[thm]{Remark}
\numberwithin{equation}{section}

\newcommand{\homework}[6]{
   \pagestyle{myheadings}
   \thispagestyle{plain}
   \newpage
   \setcounter{page}{1}
   \noindent
   \begin{center}
   \framebox{
      \vbox{\vspace{2mm}
    \hbox to 6.28in { {\bf COMP 4150 Artificial Intelligence and Machine Learning \hfill {\small (#2)}} }
       \vspace{6mm}
       \hbox to 6.28in { {\Large \hfill #1  \hfill} }
       \vspace{6mm}
       \hbox to 6.28in { {\it Instructor: {\rm #3} \hfill Name: {\rm #5}, ID: {\rm #6}} }
       %\hbox to 6.28in { {\it TA: #4  \hfill #6}}
      \vspace{2mm}}
   }
   \end{center}
   \markboth{#5 -- #1}{#5 -- #1}
   \vspace*{4mm}
}

\newcommand{\problem}[2]{~\\\fbox{\textbf{Problem #1}}\hfill #2\newline\newline}
\newcommand{\subproblem}[1]{~\newline\textbf{(#1)}}
\newcommand{\D}{\mathcal{D}}
\newcommand{\Hy}{\mathcal{H}}
\newcommand{\VS}{\textrm{VS}}
\newcommand{\solution}{~\newline\textbf{\textit{(Solution)}} }

\newcommand{\bbF}{\mathbb{F}}
\newcommand{\bbX}{\mathbb{X}}
\newcommand{\bI}{\mathbf{I}}
\newcommand{\bX}{\mathbf{X}}
\newcommand{\bY}{\mathbf{Y}}
\newcommand{\bepsilon}{\boldsymbol{\epsilon}}
\newcommand{\balpha}{\boldsymbol{\alpha}}
\newcommand{\bbeta}{\boldsymbol{\beta}}
\newcommand{\0}{\mathbf{0}}

\usepackage{amsmath}




\begin{document}
\homework{Assignment 1}{Due: 24 Mar 2019}{Prof CHEUNG Yiu Ming}{}{Matteo Azzarelli}{18432468}

\problem{Question 1}{}
\subproblem{a}
      In order to find the size of the state space we should consider that in this problem is allowed go backward, so, that it means is possible have loops.
      The blank tile can be in every positions (7), the white tiles can be in every remaining space, so, is the combination of 3 element of 6 $$C(6,3)= \frac{n!}{r!(n-r)!}=\frac{6!}{3!(6-3)!}=\frac{6!}{3!3!}=\frac{6\cdot 5\cdot 4\cdot 3\cdot 2}{3\cdot 2\cdot 3\cdot 2}= 5\cdot 4$$
      At this time we have $7 \cdot5 \cdot 4$ and the remaining 3 black tiles are forced in the remaining space, so we have a total of $$7 \cdot5 \cdot 4 = 140$$ different states.
      
\subproblem{b}
    One heuristic function h for solve this problem is h = "count number of white tiles on the right side of black tiles".
    This heuristic is admissible because the estimated cost of the cheapest solution through n, where $f(n) = g(n) + h(n)$, never overestimate the actual cost of the best solution. g(n) is the path cost from the start node to the node n and h(n) is the estimated cost of the cheapest path from n to the goal.
    
    In addition, h is also monotonic because if n and n' are two nodes where n is the parent of n', it follow $f(n) \leq f(n')$.
      
\newpage
\problem{Question 2}{}

\\
\vspace{22cm}The path with the lower costs is:
$$S \longrightarrow A \longrightarrow B \longrightarrow C \longrightarrow G , 12.4$$
\newpage
\problem{Question 3}{}

\subproblem{a}
    Convert to predicate logic:
\begin{flalign*}
    &1) \; \forall x \; Pass(x, History) \wedge Win(x, Lottery) \Rightarrow Happy(x)\\
    &2) \; \forall x \forall y \; Study(x) \lor Lucky(x) \Rightarrow Pass(x, y)\\
    &3) \; \lnot Study(Jhon) \wedge Lucky(Jhon)\\
    &4) \; \forall x \; Lucky(x) \Rightarrow Win(x, Lottery)
\end{flalign*}

Convert to CNF, eliminate implications:
\begin{flalign*}
    &1) \; \forall x \; \lnot(Pass(x, History) \wedge Win(x, Lottery)) \lor Happy(x)\\
    &2) \; \forall x \forall y \; \lnot(Study(x) \lor Lucky(x)) \lor Pass(x, y)\\
    &3) \; \lnot Study(Jhon) \wedge Lucky(Jhon)\\
    &4) \; \forall x \; \lnot Lucky(x) \lor Win(x, Lottery)
\end{flalign*}

Next step, Move \lnor inward:
\begin{flalign*}
    &1) \; \forall x \; (\lnot Pass(x, History) \lor \lnot Win(x, Lottery)) \lor Happy(x)\\
    &2) \; \forall x \forall y \; (\lnot Study(x) \wedge \lnot Lucky(x)) \lor Pass(x, y)\\
    &3) \; \lnot Study(Jhon) \wedge Lucky(Jhon)\\
    &4) \; \forall x \; \lnot Lucky(x) \lor Win(x, Lottery)
\end{flalign*}

Next, drop the qualifiers:
\begin{flalign*}
    &1) \; (\lnot Pass(x, History) \lor \lnot Win(x, Lottery)) \lor Happy(x)\\
    &2) \; (\lnot Study(x) \lor Pass(x, y)) \wedge (\lnot Lucky(x) \lor Pass(x, y))\\
    &3) \; \lnot Study(Jhon) \wedge Lucky(Jhon)\\
    &4) \; \lnot Lucky(x) \lor Win(x, Lottery)
\end{flalign*}

Now we can separate the litterals:
\begin{flalign*}
    &1) \; (\lnot Pass(x, History) \lor \lnot Win(x, Lottery)) \lor Happy(x)\\
    &2.a) \; \lnot Study(x) \lor Pass(x, y)\\
    &2.b) \; \lnot Lucky(x) \lor Pass(x, y)\\
    &3.a) \; \lnot Study(Jhon)\\
    &3.b) \; Lucky(Jhon)\\
    &4) \; \lnot Lucky(x) \lor Win(x, Lottery)
\end{flalign*}
            
Transform in INF:
\begin{flalign*}
    &1) \; \forall x \; \lnot(Pass(x, History) \wedge Win(x, Lottery)) \lor Happy(x)\\
    &2.a) \; Study(x) \Rightarrow Pass(x, y)\\
    &2.b) \; Lucky(x) \Rightarrow Pass(x, y)\\
    &3.a) \; Study(Jhon) \Rightarrow FALSE\\
    &3.b) \; TRUE \Rightarrow Lucky(Jhon)\\
    &4) \; Lucky(x) \Rightarrow Win(x, Lottery)
\end{flalign*}

\newpage
\subproblem{b}
    Proof by resolution refutation or contradiction:
    the goal is proof $Happy(John)$, so we add the clause $\lnot Happy(John)$ and we aspect the result False in order to proof our goal.
\vspace{20cm}


\problem{Question 4}{}
Can't be any consistent hypothesis for any set of positive training examples because the precision could be different, in particular there could be another decimal digit and this make it impossible.

A possible way to solve this problem could be set a limit for the maximum number of decimals in the hypothesis language.

\problem{Question 5}{}
\begin{flalign*}
S_{0} &= \left \{<\varnothing, \varnothing, \varnothing > \right \}\\
G_{0} &= \left \{<?,?,?> \right \}\\
\\
S_{1} &=\left \{<bad, high, none> \right \}\\
G_{1} &= \left \{<?,?,?> \right \}\\
\\
S_{2} &= \left \{<bad, high, none> \right \}\\
G_{2} &= \left \{<bad,?,?>,<?,high,?> \right \}\\
\\
S_{3} &= \left \{<bad, high, none> \right \}\\
G_{3} &= \left \{<bad,low,?>,<bad,?,none>,<?,high,?><?,?,none> \right \}\\
\\
S_{3} &= \left \{<bad, high, none> \right \}\\
G_{3} &= \left \{<bad,low,?>,<bad,?,none>,<?,high,?><?,?,none> \right \}\\
\\
S_{4} &= \left \{<?, high, none> \right \}\\
G_{4} &= \left \{<?,high,?><?,?,none> \right \}\\
\end{flalign*}

$$S_{4} = \left \{<?, high, none> \right \}$$
\vspace{3cm}
$$G_{4} = \left \{<?,high,?><?,?,none> \right \}$$

\end{document} 


